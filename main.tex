%%%%%%%%%%%%%%%%%%%%%%%%%%%%%%%%%%%%%%%%%
% Simple Sectioned Essay Template
% LaTeX Template
%
% This template has been downloaded from:
% http://www.latextemplates.com
%
% Note:
% The \lipsum[#] commands throughout this template generate dummy text
% to fill the template out. These commands should all be removed when 
% writing essay content.
%
%%%%%%%%%%%%%%%%%%%%%%%%%%%%%%%%%%%%%%%%%

%----------------------------------------------------------------------------------------
%	PACKAGES AND OTHER DOCUMENT CONFIGURATIONS
%----------------------------------------------------------------------------------------

\documentclass[12pt]{article} % Default font size is 12pt, it can be changed here

\usepackage{listings}
\lstset{
    basicstyle=\ttfamily,
    mathescape
}

\usepackage{hyperref}

% \usepackage{geometry} % Required to change the page size to A4
% \geometry{a4paper} % Set the page size to be A4 as opposed to the default US Letter

\usepackage{graphicx} % Required for including pictures
\usepackage{amsfonts}
\usepackage{amsmath}

\usepackage{float} % Allows putting an [H] in \begin{figure} to specify the exact location of the figure
\usepackage{wrapfig} % Allows in-line images such as the example fish picture

\usepackage{lipsum} % Used for inserting dummy 'Lorem ipsum' text into the template

\linespread{1.2} % Line spacing

%\setlength\parindent{0pt} % Uncomment to remove all indentation from paragraphs

%\graphicspath{{Pictures/}} % Specifies the directory where pictures are stored

\begin{document}

\title{Autocompletion on Wikipedia corpus (Second Try)}
\author{Wei Zhong}
\maketitle

Last time we identify the following major defects:
\begin{itemize}
\item No symbol-wise similarity, this tends to allow many semantically different expressions being represented by the first-appear one which may not be an intuitive results.
\item Only taking into account the frequency, and the frequency does not reflect popularity due to lack of query log and also the above point.
\item No structure similarity, a simple query may suggest a very complex and large expression.
\item Lack the definition for ``unique expression''.
\end{itemize}

To address the above issues, first I propose a definition to distinguish ``unique expressions''.
Basically it performs a search on existing index, and use the Approach0 similarity measurement (including structural and symbolic similarity)
to test the uniqueness of a expression with all relevant expressions in existing index.

The uniqueness is the inverse of highest similarity score (scaled from 0 to 1) between query and all relevant expressions.
Given a threshold value (currently 0.9), if any indexed expression has a similarity score larger than the threshold, it is treated as identical expressions, and the frequency will
increase onto the expression ID which represents the unique expressions.
Otherwise, it is an unique expression, and will be indexed with a separate expression ID with an initial frequency equals to one.

The downside of measuring uniqueness by a thorough search per indexing expression is that it largely increases index time, on my machine, it takes more than 3 hours to index the Wiki dataset.

On the other hand, at query time, the new version not only takes into account frequency, but also include symbol-wise/structure similarity, search level (corresponding to the nested level
of a sub-expression match).
The formula for scoring suggestions after several rounds of tuning is,
$$
\text{score} = \frac{s \cdot f \cdot l }{1 + l \cdot \log(1 + l)}
$$
where $s$ is scaled similarity score produced by original  Approach0 similarity measurement (including structural and symbolic similarity),
$l$ is search level and $f$ is the frequency of that unique expression.

The basic idea for this scoring formula is that we want to penalize search depth (so that we tend to avoid huge suggestion expression) but this penalty should not be too much and we may combine the frequency and search depth together, since a large but good suggestion (popular) can save user more typing. So it only uses a log level penalty factor at denominator while also multiplying the frequency at nominator. Furthermore, it adds structure and symbol similarity awareness by introducing $s$ onto nominator.

During implementation, I also spot a query script bug which explains why last time we get many structurally irrelevant result.
When query JSON is sent to qac daemon, the query script does not strip the document identifier (e.g. "2D\_computer\_graphics:0) so that the identifier is treated as part of LaTeX string.
See the following commit for the bug fix:
\url{https://github.com/approach0/search-engine/commit/dbbafb19232186efb25b5b509ff0e1ed4babb7fe#diff-48e4aa3dec7962c5ee797ddb21e8e22d}

\pagebreak
The concluded algorithm pseudo code is shown below:
\vspace{2cm}

\begin{lstlisting}
Function index(tex)
    result := math_expr_search(tex);
    if result is not empty:
        threshold := MAX_MATH_EXPR_SIM_SCALE * 0.9
        for r in result:
            if r.score > threshold:
                key_value_db[r.exprID].freq += 1
    else:
        operator_tree := parse(tex)
        if no parser error reported:
            cur_len := len(key_value_db)
            new_exprID := cur_len + 1
            key_value_db[new_exprID].freq := 1
            Index math expression by operator_tree

Function search(tex)
    result := math_expr_search(tex);
    for each result f := key_value_db[new_exprID].freq
    rank result by scoring formula $\frac{s \cdot f \cdot l }{1 + l \cdot \log(1 + l)}$
\end{lstlisting}

\pagebreak
\section*{Generated example}

\subsection{Query $x^2+y$}
\begin{center}
\begin{tabular}{lccc}
Suggesions $\backslash$ Query  & $x^2 $ & Frequency & Score \\
\hline
 1 &  $ O(R^{2})\,\! $ & 302 & 46.85 \\
 2 &  $ r^{2}m $ & 165 & 32.26 \\
 3 &  $ q\sim p^{2} $ & 131 & 25.61 \\
 4 &  $ f^{2}(\theta) $ & 84 & 16.42 \\
\end{tabular}
\end{center}

\begin{center}
\begin{tabular}{lccc}
Suggesions $\backslash$ Query  & $x^2+ $ & Frequency & Score \\
\hline
 1 &  $ x^{2}+y^{2}=r^{2}.\! $ & 100 & 9.80 \\
 2 &  $ \scriptstyle r=\sqrt{a^{2}+b^{2}} $ & 44 & 3.63 \\
 3 &  $ x^{2}+y^{2}=1 $ & 25 & 2.95 \\
 4 &  $ \sqrt{a^{2}+b^{2}} $ & 19 & 2.37 \\
\end{tabular}
\end{center}


\begin{center}
\begin{tabular}{lccc}
Suggesions $\backslash$ Query  & $x^2 + y$ & Frequency & Score \\
\hline
 1 &  $ f(x)=x^{3}+x $ & 11 & 1.61 \\
 2 &  $ x^{2}+px+q=0 $ & 12 & 1.41 \\
 3 &  $ f_{c}(z)=z^{2}+c $ & 9 & 1.05 \\
 4 &  $ z\mapsto\bar{z}^{2}+c\,. $ & 5 & 0.98 \\
\end{tabular}
\end{center}

\pagebreak
%%%%%%%%%%%%%%%%%%%%%%%%
%%%%%%%%%%%%%%%%%%%%%%%%
%%%%%%%%%%%%%%%%%%%%%%%%
%%%%%%%%%%%%%%%%%%%%%%%%

\subsection{Query $ax^2+b$}

\begin{center}
\begin{tabular}{lccc}
Suggesions $\backslash$ Query  & $ ax $ & Frequency & Score \\
\hline
 1 &  $ d_{BO}\!\,- $ & 2667 & 412.08 \\
 2 &  $ \mathbf{F}=m\mathbf{g} $ & 1287 & 250.84 \\
 3 &  $ x_{\mathrm{new}} $ & 829 & 102.68 \\
 4 &  $ b^{th} $ & 521 & 80.50 \\
\end{tabular}
\end{center}

\begin{center}
\begin{tabular}{lccc}
Suggesions $\backslash$ Query  & $ ax^2 $ & Frequency & Score \\
\hline
 1 &  $ I=r^{2}m $ & 96 & 18.77 \\
 2 &  $ O(VE^{2}) $ & 38 & 5.90 \\
 3 &  $ q\,=\,\tfrac{1}{2}\,\rho\,v^{2} $ & 47 & 5.50 \\
 4 &  $ P=wc^{2}\rho $ & 34 & 4.98 \\
\end{tabular}
\end{center}

\begin{center}
\begin{tabular}{lccc}
Suggesions $\backslash$ Query  & $ ax^2 + $ & Frequency & Score \\
\hline
 1 &  $ ax^{2}+bx+c=0 $ & 19 & 1.86 \\
 2 &  $ y=ax^{2}+bx+c\, $ & 38 & 1.57 \\
 3 &  $ \ p(x)=ax^{2}+bx+c $ & 14 & 1.17 \\
 4 &  $ ax^{2}+bx=c $ & 6 & 0.71 \\
\end{tabular}
\end{center}

\begin{center}
\begin{tabular}{lccc}
Suggesions $\backslash$ Query  & $ ax^2 + b $ & Frequency & Score \\
\hline
 1 &  $ ax^{2}+bx+c=0 $ & 19 & 2.23 \\
 2 &  $ y=ax^{2}+bx+c $ & 16 & 1.88 \\
 3 &  $ \ p(x)=ax^{2}+bx+c $ & 14 & 1.37 \\
 4 &  $ a+b\alpha+c\alpha^{2}, $ & 5 & 0.62 \\
 5 &  $ \sqrt{a+bz+cz^{2}} $ & 4 & 0.52 \\
\end{tabular}
\end{center}

\pagebreak
%%%%%%%%%%%%%%%%%%%%%%%%
%%%%%%%%%%%%%%%%%%%%%%%%
%%%%%%%%%%%%%%%%%%%%%%%%
%%%%%%%%%%%%%%%%%%%%%%%%

\subsection{Query $(1+ \frac 1 n)^n$}

\begin{center}
\begin{tabular}{lccc}
Suggesions $\backslash$ Query  & $ 1+ $ & Frequency & Score \\
\hline
 1 &  $ \displaystyle a_{n+1} $ & 319 & 39.91 \\
 2 &  $ (y+1) $ & 95 & 18.68 \\
 3 &  $ N=M+1\, $ & 59 & 8.71 \\
 4 &  $ m=+1 $ & 44 & 8.65 \\
 5 &  $ F(n+1) $ & 51 & 6.38 \\
\end{tabular}
\end{center}

\begin{center}
\begin{tabular}{lccc}
Suggesions $\backslash$ Query  & $ 1+\frac 1 n $ & Frequency & Score \\
\hline
 1 &  $ 1+\frac{1}{\varphi}=\varphi $ & 4 & 0.78 \\
 2 &  $ \frac{1}{r}+\frac{1}{p}=1+\frac{1}{s} $ & 3 & 0.29 \\
 3 &  $ e=\lim_{n\to\infty}\left(1+\frac{1}{n}\right)^{n}. $ & 6 & 0.27 \\
 5 &  $ \frac{n+1}{n}=1+\frac{1}{n} $ & 2 & 0.24 \\
\end{tabular}
\end{center}

\begin{center}
\begin{tabular}{lccc}
Suggesions $\backslash$ Query  & $ (1+\frac 1 n) $ & Frequency & Score \\
\hline
 1 &  $ e=\lim_{n\to\infty}\left(1+\frac{1}{n}\right)^{n}. $ & 6 & 0.32 \\
 2 &  $ (1+1/n)^{n} $ & 2 & 0.31 \\
 3 &  $  \lim_{n\to\infty}\left(1+\frac{1}{n}\right)^{n} $ & 4 & 0.29 \\
 4 &  $ P(1+1/{\epsilon})=MC $ & 2 & 0.21 \\
 5 &  $ (1+1/n)r $ & 1 & 0.20 \\
\end{tabular}
\end{center}

\begin{center}
\begin{tabular}{lccc}
Suggesions $\backslash$ Query  & $ (1+\frac 1 n)^n $ & Frequency & Score \\
\hline
 1 &  $ e=\lim_{n\to\infty}\left(1+\frac{1}{n}\right)^{n}. $ & 6 & 0.53 \\
 2 &  $  \lim_{n\to\infty}\left(1+\frac{1}{n}\right)^{n} $ & 4 & 0.47 \\
 3 &  $ \lim_{x\to+\infty}\left(1+\frac{1}{x}\right)^{x}=e $ & 1 & 0.09 \\
 4 &  $ \lim_{n\to\infty}\left(1+\frac{1}{n}\right)^{n}, $ & 1 & 0.08 \\
 5 &  $ \left(1+\frac{1}{x}\right)^{x}<e<\left(1+\frac{1}{x}\right)^{x+1} $ & 1 & 0.07 \\
\end{tabular}
\end{center}

\pagebreak
%%%%%%%%%%%%%%%%%%%%%%%%
%%%%%%%%%%%%%%%%%%%%%%%%
%%%%%%%%%%%%%%%%%%%%%%%%
%%%%%%%%%%%%%%%%%%%%%%%%

\section*{Current problem}
Mainly the index time is too long, I put a limit to the maximum number of document and directories to be read per query.
But I have not tested the new index time yet.

\end{document}
