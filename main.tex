%%%%%%%%%%%%%%%%%%%%%%%%%%%%%%%%%%%%%%%%%
% Simple Sectioned Essay Template
% LaTeX Template
%
% This template has been downloaded from:
% http://www.latextemplates.com
%
% Note:
% The \lipsum[#] commands throughout this template generate dummy text
% to fill the template out. These commands should all be removed when 
% writing essay content.
%
%%%%%%%%%%%%%%%%%%%%%%%%%%%%%%%%%%%%%%%%%

%----------------------------------------------------------------------------------------
%	PACKAGES AND OTHER DOCUMENT CONFIGURATIONS
%----------------------------------------------------------------------------------------

\documentclass[12pt]{article} % Default font size is 12pt, it can be changed here

\usepackage{geometry} % Required to change the page size to A4
\geometry{a4paper} % Set the page size to be A4 as opposed to the default US Letter

\usepackage{graphicx} % Required for including pictures

\usepackage{float} % Allows putting an [H] in \begin{figure} to specify the exact location of the figure
\usepackage{wrapfig} % Allows in-line images such as the example fish picture

\usepackage{lipsum} % Used for inserting dummy 'Lorem ipsum' text into the template

\linespread{1.2} % Line spacing

%\setlength\parindent{0pt} % Uncomment to remove all indentation from paragraphs

\graphicspath{{Pictures/}} % Specifies the directory where pictures are stored

\begin{document}

\section*{Autocompletion on Wikipedia corpus}
There are 591,468 formula in Wikipedia corpus from NTCIR-12 Formula Browser task,
132,372 structrually unique TeX expressions (consider commutativity) obtained after indexing.

\textbf{Addressing the assigned question}:
Current occurrence-of-formula may not well reflect formula popularity, in the current implementation, popularity is counted by formula frequency, however, the case study shows (see next section) suggested candidate might be not very popular according to common sense math knowledge, the two major reasons are:
\begin{itemize}
\item The Wikipedia corpus does reflect popularity very well,
fundamental design of Wikipedia is to describe one concept only in one page, so popular formula such as Pythagorean theorem is not shown too many times in the corpus. Surprisingly, only 7 times (first appearing in Algebraic\_curve:51) according to our index. This suggests a better choice of corpus would be \textbf{Q\&A website} (e.g. Math StackExchange) where a concept like Pythagorean theorem may be asked/described much more frequently than these in Wikipedia.
\item The current implementation counts formula frequency, however, many rare formula overall may appear in a single document (e.g. particularly addressing this formula) many times. But much less or few mentioned in other corpus document. One suggestion for the next step is to use \textbf{document frequency} instead as the popularity number to reflect formula frequency, such that the it would not be interfered by those document/Wiki page that particularly describe a rare math concept/formula extensively.
\end{itemize}

As for whether we should account subexpressions.
I believe in terms of QAC model, this is not desired, so think about $a^2+b^2=c^2$, if every subexpression is indexed, then in a sufficiently large corpus the following inequality holds and the frequency gap can be very large: $freq(a^2+b^2=c^2) < freq(a^2+b^2) < freq(a^2+) < freq(a^2)$. So if querying $a^2$, the desired formula $a^2+b^2=c^2$ is far less significant in terms of popularity compared to $a^2 + $, $(a^2)$ etc. It is more rare to ``jump'' directly to suggest a common formula like  $a^2+b^2=c^2$ and save user's input more than one character where in QAC the number of characters saved from typing is basically the major metric to measure effectiveness of autocomplete system. 

\section*{Generated example}

\section{Reference}
{\setlength{\parindent}{0cm}
[1] Song, Yang, Ziming Zhuang, Huajing Li, Qiankun Zhao, Jia Li, Wang-Chien Lee, and C. Lee Giles. “Real-Time Automatic Tag Recommendation.” In Proceedings of the 31st Annual International ACM SIGIR Conference on Research and Development in Information Retrieval, 515–522. SIGIR ’08. New York, NY, USA: ACM, 2008. https://doi.org/10.1145/1390334.1390423.

} \end{document}
