\documentclass[a4paper]{article}

%% Language and font encodings
\usepackage[english]{babel}
\usepackage[utf8x]{inputenc}
\usepackage[T1]{fontenc}

%% Sets page size and margins
\usepackage[a4paper,top=3cm,bottom=2cm,left=3cm,right=3cm,marginparwidth=1.75cm]{geometry}

%% Useful packages
\usepackage{amsmath}
\usepackage{graphicx}
\usepackage[colorinlistoftodos]{todonotes}
\usepackage[colorlinks=true, allcolors=blue]{hyperref}

\title{Your Paper}
\author{You}

\begin{document}
\maketitle

\section{Introduction}

Your introduction goes here! Some examples of commonly used commands and features are listed below, to help you get started. If you have a question, please use the help menu (``?'') on the top bar to search for help or ask us a question. 

\section{Some examples to get started}

\begin{figure}
\centering
\includegraphics[width=0.3\textwidth]{frog.jpg}
\caption{\label{fig:frog}This frog was uploaded via the project menu.}
\end{figure}

You can upload a .bib file containing your BibTeX entries, created with JabRef; or import your \href{https://www.overleaf.com/blog/184}{Mendeley}, CiteULike or Zotero library as a \verb|.bib| file. You can then cite entries from it, like this: \cite{greenwade93}. Just remember to specify a bibliography style, as well as the filename of the \verb|.bib|.

You can find a \href{https://www.overleaf.com/help/97-how-to-include-a-bibliography-using-bibtex}{video tutorial here} to learn more about BibTeX.

We hope you find Overleaf useful, and please let us know if you have any feedback using the help menu above --- or use the contact form at \url{https://www.overleaf.com/contact}!

\bibliographystyle{alpha}
\bibliography{sample}

\end{document}
